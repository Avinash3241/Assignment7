\documentclass[journal,twocolumn]{IEEEtran}
\usepackage[utf8]{inputenc}
\usepackage{amssymb}
\usepackage{amsmath}
\usepackage{graphicx}
\newcommand{\myvec}[1]{\ensuremath{\begin{pmatrix}#1\end{pmatrix}}}
\providecommand{\brak}[1]{\ensuremath{\left(#1\right)}}
\title{Assignment 7}
\author{NIMMALA AVINASH(CS21BTECH11039)}
\begin{document}

\maketitle

{\LARGE \textbf{Question:\\}}
\begin{large}
The events A,B and C are such that \\
\begin{align}
P(A)=P(B)&=P(C)=0.5\\
P(AB)=P(AC)&=P(BC)=P(ABC)=0.25
\end{align}

show that the zero-one random variables associated with these events are not independent;they are,however,independent in pairs.\\
\end{large}

{\LARGE \textbf{Solution:\\}}
\begin{large}
\begin{align}
P\{x_{A}=1,x_{B}=1,x_{C}=1\} = P(ABC)&=\frac{1}{4}\\
P\{x_{A}=1\} = P(A) =&\frac{1}{2}\\ P\{x_{B}=1\} = P(B) =& \frac{1}{2}\\
P\{x_{C}=1\}=P(c)=&\frac{1}{2}\\
P\{x_{A}=1,x_{B}=1,x_{C}=1\}\neq P\{x_{A}=1\}P\{x_{B}=&1\}P\{x_{C}=1\}
\end{align}
hence $ x_{A},x_{B},x_{C}$ are not independent. But\\
\begin{align}
P\{x_{A}=1,x_{B}=1\}=P(AB)=\frac{1}{4}=p\{x_{A}=1\}p\{x_{C}=1\}
\end{align}
similarly for any other combination,e.g.,\\since
$ P(A) = P(AB)+P(A\overline{B}),$we conclude that\\
$P(A\overline{B}) = P(A) - P(AB)\\$
\begin{align}
P(A\overline{B})=\frac{1}{2}-\frac{1}{4} = \frac{1}{4}\\
P(\overline{B})=1-P(B)=\frac{1}{2}\\
p\{x_{A}=1,x_{B}=0\}=P(A\overline{B})=\frac{1}{4}\\
p\{x_{B}=0\}=P(\overline{B})=\frac{1}{2} \\
\implies p\{x_{A}=1,x_{B}=0\}=p\{x_{A}=1\}p\{x_{B}=0\}
\end{align}
So,these are independent in pairs.\\
\end{large}
\end{document}